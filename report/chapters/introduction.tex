\chapter*{Introduction}

\addcontentsline{toc}{chapter}{Introduction}

\paragraph{}
Dans le milieu pétrochimique ou dans la chimie pédagogique, nous assistons à
une réelle demande d'outils de simulations d'éléments à vue moléculaire ou
atomique: une représentation graphique d'une réaction permet une meilleure
compréhension de celle-ci, et facilite l'apprentissage en complétant les
maquettes, souvent utilisées.

\paragraph{}
Pour répondre à ce besoin, dans le cadre de notre projet tutoré dans l'Unité de
Valeur TA72, nous avons entamé la conception d'un outil de simulations
graphique d'atomes sous la tutelle de Monsieur Fougères. Nous ne nous
démarrions pas le projet de zéro puisqu'il a également fait l'objet du projet
de fin de semestre de l'UV LP74, suite à laquelle nous nous sommes vus proposer
de le continuer.

\paragraph{}
Tel que laissé en LP74, l'outil affichait une représentation 3D d'un
environnement d'atomes, se déplaçant de manière aléatoire mais gérant les
collisions. L'utilisateur pouvait se déplacer dans l'environnement pour
visualiser les atomes. Quelques problèmes de performances étaient présents,
certains provoquant des crashs de l'application.
