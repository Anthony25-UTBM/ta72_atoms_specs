\chapter{Mise en oeuvre}
\label{mise_en_oeuvre}

\section{Analyse de l'existant}

\paragraph{}
Dans une optique d'amélioration de l'existant, un bilan sur l'application telle
que rendue en LP74 a été effectué.


\subsection{Application dans un unique thread}

\paragraph{}
Le dessin des atomes, la gestion de leurs déplacements mais ainsi que toute
navigation dans l'interface sont effectués dans un unique thread.

\paragraph{}
Les atomes et molécules sont stockées dans une liste chainée, parcourue de
manière continue et séquentielle par l'application pour mettre à jour leurs
coordonnées. Il est alors nécessaire que le calcul des nouvelles coordonnées
ainsi que le dessin de chaque sphère soient simples et rapides pour rester
inférieur à la durée d'une frame, afin obtenir des mouvements fluides.

\paragraph{}
De cette manière, les atomes n'étaient pas indépendants : un objet
Environnement coordonne l'ensemble, et force de lui-même les mises à jour des
informations.  Il a également comme rôle de vérifier les collisions.


\subsection{Consommation excessive de mémoire}

\paragraph{}
L'application souffre de problèmes de consommation mémoire, avec des fuites
dues à JavaFX (le framework graphique utilisé) qui amènent des lenteurs.
L'application se fige par moments, et il arrive également qu'elle se fasse
tuer par l'OOM killer du système.


\subsection{Mouvement strictement basés sur de l'aléatoire}

\paragraph{}
Les mouvements des atomes ne sont pas basés sur des attractions ou répulsions
entre éléments. L'application impose un aspect ``brouillon'' dans les vecteurs
vitesses des éléments, de façon à donner l'impression que l'environnement n'est
pas uniforme. Tout repose sur un système aléatoire, avec gestion de collisions
et inversions de vecteurs vitesses lorsqu'un élément s'approche de trop près
des bords du bac.


\subsection{Formations de molécules}

\paragraph{}
De façon à former des molécules, et contrairement à ce qui est indiqué dans la
partie liée aux mouvements, une sorte d'attraction a été mise en place. Nous ne
considérons pas que le système en place soit suffisamment exact pour qu'il ait
une réelle influence sur les mouvements des atomes.

\paragraph{}
Néanmoins, si la vitesse d'un atome est suffisamment faible et qu'il rentre en
collision avec un autre atome, ces deux atomes vont se lier. La liaison ne
repose sur rien de scientifique, dans le sens où, par exemple, les couches de
valences ne sont pas prises en compte. Aucune vérification n'est effectuée
pour savoir si les atomes peuvent réellement se lier entre eux.
